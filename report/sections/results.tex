\section{Results}
The following results were obtained for a run on the \emph{JUSUF}\footnote{\url{https://www.fz-juelich.de/en/ias/jsc/systems/supercomputers/jusuf}} cluster with $2$ nodes and $4$ tasks per node. The simulation ran for $N = \num{1 200 000}$ sweeps on lattice sizes $L\in\{32, 48, \dots, 272\}$ and the bootstrap parameters were $A = N$ and $B = \num{200 000}$.

\subsection{Observables}\label{sec:observables}
\subsubsection{Energy per Spin}\label{sec:energy_per_spin}
As the total energy of the system as defined in~\cref{eq:hamiltonian} scales with the number of lattice sites, it is often more insightful to observe the energy per spin
\begin{equation}\label{eq:energy_per_spin}
	E = - \frac{1}{L^2} \sum_{\langle i,j \rangle}{\cos{(\Delta \theta)}}
\end{equation}
which has been plotted in~\cref{fig:energy_per_spin}.

For low temperatures we have a quasi-ordered state where the spins mostly align. Just like in the Ising model the energy per spin is $-2.0$ when extrapolating to $T=0$. With increasing temperature the energy slowly vanishes until it asymptotically approaches $0$ for big $T$.
\begin{figure}[ht]
	\centering
	\includegraphics[width=0.8\textwidth]{Energy.pdf}
	\caption[Temperature dependence of the energy per spin $E$]{Plot of the temperature dependence of the energy per spin $E$ (\cref{eq:energy_per_spin}) for lattice sizes $L\in\{32, 48, \dots, 272\}$. For small $T$ the energy tends to $\num{-2}$ while for big temperatures the energy asymptotically approaches $\num{0}$.}
	\label{fig:energy_per_spin}
\end{figure}

\subsubsection{Magnetization per Spin}\label{sec:magnetization_per_spin}
The absolute total magnetization per spin for the system is given by
\begin{equation}\label{eq:magnetzation_per_spin}
	{\lvert M \rvert}^2 = \frac{1}{L^2} \left( \left( \sum{\cos{\theta_i}} \right)^2 + \left( \sum{\sin{\theta_i}} \right)^2 \right)
\end{equation}
and has been plotted in~\cref{fig:magnetization_per_spin}.

For low temperature the magnetzation tends to $\num{1}$ which confirms the existance of a quasi-ordered low temperature state. With increasing temperature the magnetization decreases steadily at first and then rapdily until slowly levels out in the unordered high temperature state. One can also see  some finite size effects. For increasing lattice sizes the form of the curve becomes more pronounced.
\begin{figure}[!htb]
	\centering
	\includegraphics[width=0.8\textwidth]{Magnetization.pdf}
	\caption[Temperature dependence of the magnetzation per spin ${\lvert M \rvert}^2$]{Plot of the temperature dependence of the magnetization per spin ${\lvert M \rvert}^2$ (\cref{eq:magnetzation_per_spin}) for lattice sizes $L\in\{32, 48, \dots, 272\}$. For small $T$ the magnetization tends to $1$ which indicates an ordered state. For big temperatures the magnetzation goes to $0$ which indicates a unordered state.}
	\label{fig:magnetization_per_spin}
\end{figure}

\subsubsection{Specific Heat per Spin}\label{sec:specific_heat_per_spin}
The specific heat of the system can be found by evaluating
\begin{equation}\label{eq:specific_heat_per_spin}
	C_V = \frac{\langle E^2 \rangle - {\langle E \rangle}^2}{T^2}
\end{equation}
and has bee plotted logarithmically in~\cref{fig:specific_heat}.

The specific heat is small for low temperatures and even smaller for big temperatures. Near the critical temperature there is a small peak which shrinks for increasing lattice sizes. The position of this peak moves towards smaller temperatures when the lattice size is increased. 
\begin{figure}[!htb]
	\centering
	\includegraphics[width=0.8\textwidth]{Specific_Heat.pdf}
	\caption[Temperature dependence of the specifc heat per spin $C_V$]{Plot of the temperature dependence of the specific heat per spin $C_V$ (\cref{eq:specific_heat_per_spin}) for lattice sizes $L\in\{32, 48, \dots, 272\}$. The specific heat generally shrinks with increasing lattice size and tends to $0$ for very small and big temperatures. There is peak near the critical temperature $T_C$ which shifts to the left for increasing lattice sizes.}
	\label{fig:specific_heat}
\end{figure}

\subsubsection{Magnetic Susceptibility per Spin}\label{sec:magnetic_susceptibility_per_spin}
The mangetic susceptibility of the system can be found by evaluating
\begin{equation}\label{eq:magnetic_susceptibility_per_spin}
	\chi = \frac{\langle M^2 \rangle - {\langle M \rangle}^2}{T}
\end{equation}
and has bee plotted logarithmically in~\cref{fig:magnetic_susceptibility}.

The magnetic susceptibility is small for low and high temperatures. There exists a peak near the critical temperature which slowly moves left and becomes narrower for increasing lattice sizes while the magnitude of the peak slowly shrinks.
\begin{figure}[!htb]
	\centering
	\includegraphics[width=0.8\textwidth]{Magnetic_Susceptibility.pdf}
	\caption[Temperature dependence of the mangetic susceptibility per spin $\chi$]{Plot of the temperature dependence of the magnetic susceptibility per spin $\chi$ (\cref{eq:magnetic_susceptibility_per_spin}) for lattice sizes $L\in\{32, 48, \dots, 272\}$. For low and high temperatures the magnetic susceptibility tends towards $0$. Near the critical temperature there is a peak which moves left and gets narrower for increasing lattice sizes.}
	\label{fig:magnetic_susceptibility}
\end{figure}

\subsection{Critical Temperature \texorpdfstring{$T_C$}{T}}\label{sec:critical_temperature}
\begin{figure}
	\centering
	\includegraphics[width=0.8\textwidth]{Critical_Temperature.pdf}
	\caption[Obtaining the critical temperature $\texorpdfstring{T_C}{T}$ by plotting the temperature $\texorpdfstring{T}{T}$ at $\texorpdfstring{\chi_\text{max}}{the magnetic susceptibility is maximum}$ against $\texorpdfstring{(\ln{L})^{-2}}{inverse logarithmic squared lattice size}$]{Plotting the temperature from~\cref{sec:magnetic_susceptibility_per_spin} at which $\chi$ is maximal against $\ln{L})^{-2}$ yields a linear correlation. The data was fitted using linear regression model using a least square approach. The uncertainties were estimated using a bootstrap approach. On the right the distribution of intercepts from the bootstrap resampling is plotted.}
	\label{fig:critical_temperature}
\end{figure}
To find the critical temperature where the the KT phase transition takes place one can use the magnetic susceptibility from~\cref{sec:magnetic_susceptibility_per_spin}. As shown by~\citet{shifted} there exists a shifted temperature $T^*$ which asymptotically approaches the critical temperature $T_C$ for an infinite lattice
\begin{equation}\label{eq:shifted_temperature}
	T^*(L) \approx T_C + \frac{\pi^2}{4c (\ln{L})^2}
\end{equation}
\cite[eq. 3]{shifted}. To confirm the correlation of lattice size and shifted temperature we now take the maximum magnetic susceptibility $\chi_\text{max}$ per lattice size from~\cref{fig:magnetic_susceptibility} and plot the temperature $T$ where it occurs against $(\ln{L})^{-2}$ (\cref{fig:critical_temperature}).

We fitted an ordinary linear regression model using the least square method against our measurements. This fits our data well and confirms the correlation of lattice size and shifted temperature. To estimate our uncertainties we used a parametric bootstrap approach by resampling $\num{10 000}$ times from our measurements and redoing our linear regression model each time. As seen on the right side in~\cref{fig:magnetic_susceptibility} the distribution of intercepts $b$ follows a gaussian distribution and therefore the CLT. Taking the standard deviation as the $\SI{95}{\percent}$ confidence band we get our estimate for the critical temperature
\begin{equation}
	T_C = \num{0.8941(57)}.
\end{equation}
We can now compare our results with some literature values
\begin{itemize}
	\item In~\citet{literature_gpu} the authors used a GPU based Monte Carlo approach to estimate the critical temperature to $T_C=\num{0.8935(1)}$.
	\item In~\citet{literature_cpu} the authors used a CPU based Monte Carlo approach to estimate the critical temperature to $T_C=\num{0.89213(10)}$.
	\item A theoretical transfer matrix approach employed by~\cite{literature_theo} lead to a critical temperature of $T_C \approx \num{0.8916}$.
\end{itemize}
As these estimates lie within the confidence interval of our simulation we may conclude that our simulation was a success. It is of note that our uncertainty is much larger than those of others as we were more constrained by our computational resources and time available.

\subsection{Vortex Unbinding}\label{sec:vortex_unbinding}
The visualize the vortex unbinding at low temperature we did a singular run were we first heated up a $64 \times 64$ lattice  to $T = \num{1.5}$ and let it thermalize there for $\num{100 000}$ sweeps. The temperature was then lowered back down to $T = \num{0.05}$ in $90$ steps with $20$ sweeps of thermalization at each temperature. At $T = \num{0.05}$ the system was given $\num{90 000}$ sweeps to let the vortices unbind. An animation of that procedure may be viewed here\footnote{\url{https://www.youtube.com/watch?v=Gi8mL_0HFxs}} while some of the key frames are shown in~\cref{fig:vortex_unbinding} of~\cref{app:vortex_unbinding}.

Below the critical temperature $T_C$ bound vortex / anti-vortex pairs appear. Is the temperature further lowered the pairs come closer together until they annihilate. The lattice is left in an quasi-ordered low temperature state.

\subsection{Performance}
\subsubsection{Scheduling}
Test
\begin{figure}
	\centering
	\includegraphics[width=0.8\textwidth]{Schedueling.pdf}
	\caption{Scheduling}
\end{figure}

\subsubsection{Lattice Scaling}
Test
\begin{figure}
	\centering
	\includegraphics[width=0.8\textwidth]{Time_Scaling.pdf}
	\caption{Time scaling}
\end{figure}

\subsubsection{Critical slowing down}
Test
\begin{figure}
	\centering
	\includegraphics[width=0.8\textwidth]{Critical_Slowing_Down.pdf}
	\caption{Critical slowing down}
\end{figure}
