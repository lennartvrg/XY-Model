\section{Source Code}\label{app:source_code}
Test

\section{Vortex Unbinding}\label{app:vortex_unbinding}
\begin{figure}[H]
	\centering
	\begin{subfigure}[h]{0.45\textwidth}
		\centering
		\includegraphics[width=\textwidth]{frames/output_001.png}
		\caption{At $T=\num{1.5}$ we are in the high temperature state were there are now vortex pairs and the spins are unordered.}
	\end{subfigure}
	~
	\begin{subfigure}[h]{0.45\textwidth}
		\centering
		\includegraphics[width=\textwidth]{frames/output_120.png}
		\caption{After cooling the system down to $T=\num{0.05}$ there are now two bound vortex pairs and the rest of the spins are aligned.}
	\end{subfigure}
	\begin{subfigure}[h]{0.45\textwidth}
		\centering
		\includegraphics[width=\textwidth]{frames/output_130.png}
		\caption{After some sweeps at $T=\num{0.05}$ the left vortex pair has annihilated while the bottom one has come closer together.}
	\end{subfigure}
	~
	\begin{subfigure}[h]{0.45\textwidth}
		\centering
		\includegraphics[width=\textwidth]{frames/output_160.png}
		\caption{After the second vortex / anti-vortex pair has annihilated we are left with an metastable lattice with ordered spins.}
	\end{subfigure}
	\caption[Vortex unbiding of vortex / antivortex pairs at low temperatures.]{Simulating the process of vortex unbinding by bringing a thermalized high temperature unordered lattice slowly to low temperatures as described in~\cref{sec:vortex_unbinding}. Below the critical temperature $T_C$ bound vortex / anti-vortex pairs appear and slowly annihilate at very low temperatures. The lattice is left in a quasi-ordered low temperature state.}
	\label{fig:vortex_unbinding}
\end{figure}