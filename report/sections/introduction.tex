\section{Introduction}
The perhaps most well known model for the magnetization of lattice structures is the Ising spin model. It describes the total magnetization of a lattice as the superposition of all spins $\sigma_i = \pm 1$ on the lattice.

\paragraph{XY model}
One can now go and generalize the problem from a $\mathbb{Z}_2$ symmetry to a continous $U(2)$ symmetry. This model is called the XY model and describes the spins as two dimensional vectors on the unit circle
\begin{equation}\label{eq:hamiltonian}
	\sigma_i = \begin{pmatrix}
		\cos{\theta_i} \\ \sin{\theta_i}
	\end{pmatrix}
\end{equation}
parametrized by the angle $\theta_i \in [0,2\pi)$. The Hamiltonian for this system is thus given as
\begin{equation}\label{eq:partition}
	H = -J \sum_{\langle i,j \rangle}{s_i \cdot s_j} = -J \sum_{\langle i,j \rangle}{\cos{(\Delta \theta)}}
\end{equation}
\citet[p. 1190, eq. (42)]{kt} where $\Delta \theta = \theta_i - \theta_j$ is the angle between two spins and $J > 0$ the interaction strength. The $\langle i,j \rangle$ notation is used to indicate a nearest neighbour approximation. The partition sum for such a system is given by
\begin{equation}
	Z = \sum{\exp{(-\beta H)}}
\end{equation}
where $\beta = (k_B T)^{-1}$ and $k_B$ being the Boltzmann constant.  For the remainder of the project we set $J = \SI{1}{\joule} k_B^{-1}$ such that $\beta = T^{-1}$ is dimensionless.

\paragraph{KT phase transition}
Unlike the Ising model, the XY model does not have a 2nd order phase transition \textquote[{\citet[p. 1190]{kt}}]{as the ground state is unstable against low-energy spin-wave excitations}. It does however have something we call KT phase transition where below a critical temperature $T_C$ a metastable state can exist. These metastable states are \textquote[{\citet[p. 1190]{kt}}]{corresponding to vortices which are closely bound in pairs} and which come closer together with decreasing temperature until they pair annihilate.