\section{Introduction}
The perhaps most well known model for the magnetization of lattice structures is the Ising spin model. It describes the total magnetization of a lattice as the superposition of all the spins $\sigma_i = \pm 1$ on the lattice.

One can now go and generalize the problem not only to a $\mathbb{Z}_2$ symmetry but to a continous $U(2)$ symmetry. This model is called the XY model and describes the spins as two dimensional vectors on the unit circle
\begin{equation}\label{eq:hamiltonian}
	\sigma_i = \begin{pmatrix}
		\cos{\theta_i} \\ \sin{\theta_i}
	\end{pmatrix}
\end{equation}
parametrized by the angle $\theta_i \in [0,2\pi)$. The Hamiltonian for this system is thus given as
\begin{equation}\label{eq:partition}
	H = - \sum_{\langle i,j \rangle}{s_i \cdot s_j} = - \sum_{\langle i,j \rangle}{\cos{(\Delta \theta)}}
\end{equation}
where $\Delta \theta = \theta_i - \theta_j$ is the angle between two spins. The partition sum for such a system is given by
\begin{equation}
	Z = \sum{\exp{(-\beta H)}}
\end{equation}
where $\beta = \frac{1}{k_B T}$.