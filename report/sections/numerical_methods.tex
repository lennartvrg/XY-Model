\section{Numerical Methods}
To get an estimate for the critical temperature and make the vortex pairs visible for the 2D lattice with periodic boundary conditions (topologically a Torus), we will use numeric Monte Carlo simulations running on a cluster of CPU nodes. \emph{Rust}\footnote{\url{https://www.rust-lang.org/}} was chosen as a fitting programming language for this project, as we want to parallelize the computational effort and the \emph{Rust} ownership model makes thread-safe access to data manageable. Detailed instructions on how to obtain and use the source code are found in~\cref{app:source_code}.

\subsection{Monte Carlo Metropolis-Hastings Algorithm}\label{sec:metropolis_hastings}
The fundamental problem we need to solve is that we want to generate configurations of our system with probability $P(\sigma) = \exp{(-\beta H[\sigma])}/Z$. The problem is that the partition sum $Z$ is not easy to calculate. As shown by~\citet{metropolis}, it is sufficient to calculate the ratio of probabilities instead
\begin{equation}
	\frac{P(y)}{P(x)} = \exp{(-\beta(H[y] - H[x]))} = \exp{(-\beta \Delta H)}.
\end{equation}
This generates a Markov chain of dependent configurations. The overall procedure for the lattice is as follows:
\begin{enumerate}
	\item Calculate $E$ and $M$ for the initial lattice configuration.
	\item Perform a lattice sweep by iterating over all lattice sites. For every site $i$ do:
	\begin{enumerate}
		\item Propose a new angle $\theta_i \in [0,2\pi)$ from a uniform distribution.
		\item Calculate $\Delta H = \Delta E$ and $\Delta M$ for the proposed new state.
		\item Accept or reject state with probability $P = \min{(1, \exp{(-\beta\Delta H)})}$.
	\end{enumerate}
	\item Update the observables $E \mathrel{{+}{=}} \Delta E$ and $M \mathrel{{+}{=}} \Delta M$ and add them to the result set.
	\item Repeat from 2. for a total of $N$ sweeps.
\end{enumerate}
The implementation is so that there is a \textit{Lattice} trait and the Metropolis-Hastings algorithm is implemented for all structs implementing that trait.

\subsection{Autocorrelation}
The configurations obtained from that procedure are dependent on the configurations preceding them.  To get a measure for that correlation one can introduce the autocorrelation
\begin{equation}
	C(\Delta t) = \langle(O(t) - \mu_O)(O(t + \Delta t) - \mu_O) \rangle
\end{equation}
(\citet[eq. (41)]{bootstrap}) and, to get a size invariant measure, the normalized autocorrelation
\begin{equation}
	\Gamma(\Delta t) = \frac{C(\Delta t)}{C(0)}
\end{equation}
(\citet[eq. (43)]{bootstrap}) with $\Delta t$ being the step between samples, $O$ an observable and $\mu_O$ the mean of that observable. This measure decays exponentially with the timescale $\tau$ being the point where samples are no longer correlated
\begin{equation}
	\tau = \frac{1}{2} + \sum^{T}_{t=1}{\Gamma(t)}
\end{equation}
(\citet[eq. (46)]{bootstrap}) where we cut off the summation when it first crosses zero. In the implementation, we use a Discrete Fourier Transform to accelerate this step.

\subsection{Thermalization and Blocking}\label{sec:blocking}
To get iid samples we discard the first $\lceil 3\tau \rceil$ values in a thermalization step. We then do a blocking step where we put the remaining samples into chunks of size $\lceil \tau \rceil$ and calculate the mean over each chunk. All further processing will use these blocked samples which are now iid.

\subsection{Bootstrap Analysis}\label{sec:bootstrap}
Since the Metropolis-Hastings algorithm scales with the number of lattice sites, it is computationally impractical to run the Metropolis-Hastings algorithm for long timescales. Given that we have an observable whose iid samples follow a Gaussian distribution and we already have \enquote{enough} iid samples which cover enough of the possible configuration space, we can then use bootstrapping to generate more samples (\citet{bootstrap}).
\begin{enumerate}
	\item Collect $B$ intermediate means by repeating the following:
	\begin{enumerate}
		\item Take $A$ random samples from the blocked samples obtained in~\cref{sec:blocking} with replacement.
		\item Calculate the mean of those samples and add the result to the set of intermediate means.
	\end{enumerate}
	\item Calculate the final mean and the sample standard deviation of those $B$ intermediate means.
\end{enumerate}

\subsection{SIMD Instructions}
Numeric calculations on the lattice such as calculating the (delta) energy use SIMD instructions. As every lattice site has four neighbours we may calculate all bonds at once as an optimization.

\subsection{Temperature Scanning}\label{sec:temperature_scanning}
The expectation is that there is an interesting area near the critical temperature and a more trivial area far from it. Using the following procedure we can zoom into the interesting area of the magnetic susceptibility which is expected to peak near the critical temperature.
\begin{enumerate}
	\item Divide the temperature interval $T \in (0.0, 3.0]$ into $64$ steps with step size $\Delta T$.
	\item Run the Metropolis-Hastings algorithm on those $64$ discrete temperatures on one thread per value. Increment the iteration count and exit if needed.
	\item Find the temperature $T$ where the magnetic susceptibility is maximal.
	\item Divide the new temperature interval $T \in (T_\text{max} - 3\Delta T, T_\text{max} + 3\Delta T]$ into $64$ steps with step size $\Delta T$ and go to 2.
\end{enumerate}


\subsection{Distributed Computing}\label{sec:distributed_computing}
The temperature scanning described in~\cref{sec:temperature_scanning} allows us to distribute the simulation across $64$ threads on a single node. As we have access to the \emph{FZ Jülich} \emph{SLURM} cluster for this project, it would be ideal to distribute the computational effort across multiple nodes.

All nodes have access to a fast network share which can be used to distribute work between nodes. We chose to use a \emph{SQLite}\footnote{\url{https://www.sqlite.org/}} database to store our results on disk as \emph{SQLite} is ACID compliant which makes it safe to access over a network share. The procedure now is as follows:
\begin{enumerate}
	\item When invoking the executable with \mintinline{bash}{srun} the user can specify which lattice sizes should be simulated.
	\item Every process tries to insert these lattice sizes in the \textit{allocations} table. The insertion process ignores duplicates so that every lattice size is only added once.
	\item During the simulation stage, every process can query the database for a lattice size (in descending order) that is not yet handled by another process. The database's ACID compliance ensures that this step is atomic.
	\item The \mintinline{bash}{srun} process simulates the lattice size and writes the result into the \textit{results} table.
\end{enumerate}
As an additional integrity measure unique constraints on the tables ensure that every lattice size and temperature combination is only inserted once. The complete database schema can be seen in~\cref{fig:schema} of~\cref{app:schema}.

